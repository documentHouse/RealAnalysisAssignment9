%\documentclass[11pt,reqno]{amsart}
\documentclass[11pt,reqno]{article}
\usepackage[margin=.8in, paperwidth=8.5in, paperheight=11in]{geometry}
%\usepackage{geometry}                % See geometry.pdf to learn the layout options. There are lots.
%\geometry{letterpaper}                   % ... or a4paper or a5paper or ... 
%\geometry{landscape}                % Activate for for rotated page geometry
%\usepackage[parfill]{parskip}    % Activate to begin paragraphs with an empty line rather than an indent7
\usepackage{graphicx}
\usepackage{pstricks}
\usepackage{amssymb}
\usepackage{epstopdf}
\usepackage{amsmath}
\usepackage{subfigure}
\usepackage{caption}
\pagestyle{plain}
%\renewcommand{\topfraction}{0.3}
%\renewcommand{\bottomfraction}{0.8}
%\renewcommand{\textfraction}{0.07}
\DeclareGraphicsRule{.tif}{png}{.png}{`convert #1 `dirname #1`/`basename #1 .tif`.png}

\title{Real Analysis $\mathbb{I}$: \\ Assignment 9}
\author{Andrew Rickert}
\date{Started: October 17, 2011 \\ \hspace{1pt} Ended: October ??,  2011}                                           % Activate to display a given date or no date

\begin{document}
\maketitle


% Page 1
\begin{flushleft} 
\textbf{Class 18.100B} - Problem 1\\
\rule{500pt}{1pt}\\
\end{flushleft} 

We are given two sequences of functions $f_n(x) = \frac{1}{nx + 1}$ and $g_n(x) = \frac{x}{nx+1}$ for $n \in \mathbb{N}$. We would like to show that $g_n$ converges uniformly on (0,1) but $f_n$ only converges pointwise.

To show that $f_n$ converges pointwise to 0 we only need to note that for each $x \in (0,1)$ we have the following relationship:
\[ f_n = \frac{1}{nx + 1} \le \frac{1}{nx}\]
Since it is know that $\lim_{n \to \infty} c s_n = c s$ assuming that $\lim_{n \to \infty}s_n = s$ we have that $\lim_{n \to \infty} f_n = 0$ for each $x$. To show uniform convergence on the other hand we need to find an $\epsilon$ such that $|f_n| \le \epsilon$ for \emph{all} $x \in (0,1)$. If we let $\epsilon = \frac{1}{4}$ then $f_N(\frac{1}{N}) = \frac{1}{2}$ so there is always a $x = \frac{1}{N}$ that violates the requirement for uniform continuity.

\indent We only need to show uniform convergence for $g_n$ since this implies pointwise convergence as well. As the following calculation shows
\[ g_n = \frac{x}{nx+1} \le \frac{x}{nx} = \frac{1}{n} \]
we can take $M_n = \frac{1}{n}$ in the Weierstrauss M-test which gives us the uniform convergence of $g_n$.

\vspace{15pt}
\begin{flushleft} 
\textbf{Class 18.100B} - Problem 2\\
\rule{500pt}{1pt}\\
\end{flushleft} 


\vspace{15pt}
\begin{flushleft} 
\textbf{Class 18.100B} - Problem 3\\
\rule{500pt}{1pt}\\
\end{flushleft} 


 
\vspace{15pt}
\begin{flushleft} 
\textbf{Class 18.100B} - Problem 4\\
\rule{500pt}{1pt}\\
\end{flushleft} 


\vspace{15pt}
\begin{flushleft} 
\textbf{Class 18.100B} - Problem 5\\
\rule{500pt}{1pt}\\
\end{flushleft} 


\vspace{15pt}
\begin{flushleft} 
\textbf{Class 18.100B} - Problem 6\\
\rule{500pt}{1pt}\\
\end{flushleft} 


\vspace{15pt}
\begin{flushleft} 
\textbf{Class 18.100B} - Problem 7\\
\rule{500pt}{1pt}\\
\end{flushleft} 

%\vspace{15pt}
%\begin{flushleft} 
%\textbf{Class 18.100B} - Extra Problem 1\\
%\rule{500pt}{1pt}\\
%\end{flushleft} 


\end{document}  