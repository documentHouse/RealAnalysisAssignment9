%\documentclass[11pt,reqno]{amsart}
\documentclass[11pt,reqno]{article}
\usepackage[margin=.8in, paperwidth=8.5in, paperheight=11in]{geometry}
%\usepackage{geometry}                % See geometry.pdf to learn the layout options. There are lots.
%\geometry{letterpaper}                   % ... or a4paper or a5paper or ... 
%\geometry{landscape}                % Activate for for rotated page geometry
%\usepackage[parfill]{parskip}    % Activate to begin paragraphs with an empty line rather than an indent7
\usepackage{graphicx}
\usepackage{pstricks}
\usepackage{amssymb}
\usepackage{epstopdf}
\usepackage{amsmath}
\usepackage{subfigure}
\usepackage{caption}
\pagestyle{plain}
%\renewcommand{\topfraction}{0.3}
%\renewcommand{\bottomfraction}{0.8}
%\renewcommand{\textfraction}{0.07}
\DeclareGraphicsRule{.tif}{png}{.png}{`convert #1 `dirname #1`/`basename #1 .tif`.png}

\title{Real Analysis $\mathbb{I}$: \\ Assignment 9}
\author{Andrew Rickert}
\date{Started: October 17, 2011 \\ \hspace{1pt} Ended: October ??,  2011}                                           % Activate to display a given date or no date

\begin{document}
\maketitle


% Page 1
\begin{flushleft} 
\textbf{Class 18.100B} - Problem 1\\
\rule{500pt}{1pt}\\
\end{flushleft} 

We are given two sequences of functions $f_n(x) = \frac{1}{nx + 1}$ and $g_n(x) = \frac{x}{nx+1}$ for $n \in \mathbb{N}$. We would like to show that $g_n$ converges uniformly on (0,1) but $f_n$ only converges pointwise.

To show that $f_n$ converges pointwise to 0 we only need to note that for each $x \in (0,1)$ we have the following relationship:
\[ f_n = \frac{1}{nx + 1} \le \frac{1}{nx}\]
Since it is know that $\lim_{n \to \infty} c s_n = c s$ assuming that $\lim_{n \to \infty}s_n = s$ we have that $\lim_{n \to \infty} f_n = 0$ for each $x$. To show uniform convergence on the other hand we need to find an $\epsilon$ such that $|f_n| \le \epsilon$ for \emph{all} $x \in (0,1)$. If we let $\epsilon = \frac{1}{4}$ then $f_N(\frac{1}{N}) = \frac{1}{2}$ so there is always a $x = \frac{1}{N}$ that violates the requirement for uniform continuity.

\indent We only need to show uniform convergence for $g_n$ since this implies pointwise convergence as well. As the following calculation shows
\[ g_n = \frac{x}{nx+1} \le \frac{x}{nx} = \frac{1}{n} \]
we can take $M_n = \frac{1}{n}$ in the Weierstrauss M-test which gives us the uniform convergence of $g_n$.

\vspace{15pt}
\begin{flushleft} 
\textbf{Class 18.100B} - Problem 2\\
\rule{500pt}{1pt}\\
\end{flushleft} 

We are given the function sequence $f_n(x) = \frac{x}{1+nx^2}$ for $x \in \mathbb{R}$ and $n \in \mathbb{N}$. We need to find or show the following about $f_n$:\\
1) Find the limit function $f$ of $f_n$\\
2) Find the limit function $g$ of $f'_n$\\
3) Show $f'(x)$ exists for all $x \in \mathbb{R}$ but $f'(0) \neq g(0)$\\
4) Find the values of $x$ such that $f'(x) = g(x)$\\
5) Find the subintervals of $\mathbb{R}$ where $f_n \to f$ uniformly\\
6) Find the subintervals of $\mathbb{R}$ where $f'_n \to g$ uniformly\\

\noindent 1) We break up the pointwise convergence of $f_n$ into the subintervals $(-\infty,0),\{0\},(0,\infty)$. We have $f_n(0) = 0$ for all $n \in \mathbb{N}$ so the limit is $f(0) = 0$. For the subinterval $(0,\infty)$ we have the following 
\[ f_n = \frac{x}{1+ nx^2} \le \frac{x}{nx^2} = \frac{1}{nx}\]
so we have $\lim_{n \to \infty} f_n \to 0$ for $x \in (0,\infty)$. For $x \in (-\infty,0)$ we have \[ \lim_{n \to \infty} f_n(x) = - \lim_{n \to \infty} f_n(-x) \to 0 \] since $-x \in (0,\infty)$.

\noindent 2) We start by calculating $f'_n$ as 
\[ f'_n = \frac{1-nx^2}{(1+nx^2)^2} \]
It is clear from substitution that $f_n(0) = 1$ for all $n \in \mathbb{N}$. The following calculation shows the value of $g$ for $x \in \mathbb{R}$ \textbackslash\{0\}:

\begin{eqnarray*}
\lim_{n \to \infty}  \frac{1-nx^2}{(1+nx^2)^2} &=& \lim_{n \to \infty} \frac{1}{(1+nx^2)^2}  -  \lim_{n \to \infty} \frac{nx^2}{(1+nx^2)^2} \\
&=& \lim_{n \to \infty} \frac{1}{(1+nx^2)^2}  -  \lim_{n \to \infty} \frac{nx^2}{1+2n x^2+n^2 x^4}\\
&=& 0 - 0 = 0
\end{eqnarray*}

\noindent So we have $g(0) = 1$ and $g(x) = 0$ for $x \in \mathbb{R}$ \textbackslash\{0\}.

\noindent 3) Since $f(x) = 0$ for all $x \in \mathbb{R}$ we have $f'(x) = 0$ for all $x$. Because $g(0) = 1$ as was shown in part 2 we have $f'(0) \neq g(0)$.

\noindent 4) $x = 0$ is the only point at which $f'$ and $g$ disagree so $f' = g$ for $x \in \mathbb{R}$ \textbackslash \{0\}.

\noindent 5) 

\vspace{15pt}
\begin{flushleft} 
\textbf{Class 18.100B} - Problem 3\\
\rule{500pt}{1pt}\\
\end{flushleft} 


 
\vspace{15pt}
\begin{flushleft} 
\textbf{Class 18.100B} - Problem 4\\
\rule{500pt}{1pt}\\
\end{flushleft} 


\vspace{15pt}
\begin{flushleft} 
\textbf{Class 18.100B} - Problem 5\\
\rule{500pt}{1pt}\\
\end{flushleft} 


\vspace{15pt}
\begin{flushleft} 
\textbf{Class 18.100B} - Problem 6\\
\rule{500pt}{1pt}\\
\end{flushleft} 


\vspace{15pt}
\begin{flushleft} 
\textbf{Class 18.100B} - Problem 7\\
\rule{500pt}{1pt}\\
\end{flushleft} 

%\vspace{15pt}
%\begin{flushleft} 
%\textbf{Class 18.100B} - Extra Problem 1\\
%\rule{500pt}{1pt}\\
%\end{flushleft} 


\end{document}  