%\documentclass[11pt,reqno]{amsart}
\documentclass[11pt,reqno]{article}
\usepackage[margin=.8in, paperwidth=8.5in, paperheight=11in]{geometry}
%\usepackage{geometry}                % See geometry.pdf to learn the layout options. There are lots.
%\geometry{letterpaper}                   % ... or a4paper or a5paper or ... 
%\geometry{landscape}                % Activate for for rotated page geometry
%\usepackage[parfill]{parskip}    % Activate to begin paragraphs with an empty line rather than an indent7
\usepackage{graphicx}
\usepackage{pstricks}
\usepackage{amssymb}
\usepackage{epstopdf}
\usepackage{amsmath}
\usepackage{subfigure}
\usepackage{caption}
\pagestyle{plain}
%\renewcommand{\topfraction}{0.3}
%\renewcommand{\bottomfraction}{0.8}
%\renewcommand{\textfraction}{0.07}
\DeclareGraphicsRule{.tif}{png}{.png}{`convert #1 `dirname #1`/`basename #1 .tif`.png}

\title{Real Analysis $\mathbb{I}$: \\ Assignment 9}
\author{Andrew Rickert}
\date{Started: October 17, 2011 \\ \hspace{1pt} Ended: December 31,  2011}                                           % Activate to display a given date or no date

\begin{document}
\maketitle


% Page 1
\begin{flushleft} 
\textbf{Class 18.100B} - Problem 1\\
\rule{500pt}{1pt}\\
\end{flushleft} 

We are given two sequences of functions $f_n(x) = \frac{1}{nx + 1}$ and $g_n(x) = \frac{x}{nx+1}$ for $n \in \mathbb{N}$. We would like to show that $g_n$ converges uniformly on (0,1) but $f_n$ only converges pointwise.

To show that $f_n$ converges pointwise to 0 we only need to note that for each $x \in (0,1)$ we have the following relationship:
\[ f_n = \frac{1}{nx + 1} \le \frac{1}{nx}\]
Since it is know that $\lim_{n \to \infty} c s_n = c s$ assuming that $\lim_{n \to \infty}s_n = s$ we have that $\lim_{n \to \infty} f_n = 0$ for each $x$. To show uniform convergence on the other hand we need to find an $\epsilon$ such that $|f_n| \le \epsilon$ for \emph{all} $x \in (0,1)$. If we let $\epsilon = \frac{1}{4}$ then $f_N(\frac{1}{N}) = \frac{1}{2}$ so there is always a $x = \frac{1}{N}$ that violates the requirement for uniform continuity.

\indent We only need to show uniform convergence for $g_n$ since this implies pointwise convergence as well. As the following calculation shows
\[ g_n = \frac{x}{nx+1} \le \frac{x}{nx} = \frac{1}{n} \]
we can take $M_n = \frac{1}{n}$ in the Weierstrauss M-test which gives us the uniform convergence of $g_n$.

\vspace{15pt}
\begin{flushleft} 
\textbf{Class 18.100B} - Problem 2\\
\rule{500pt}{1pt}\\
\end{flushleft} 

We are given the function sequence $f_n(x) = \frac{x}{1+nx^2}$ for $x \in \mathbb{R}$ and $n \in \mathbb{N}$. We need to find or show the following about $f_n$:\\
1) Find the limit function $f$ of $f_n$\\
2) Find the limit function $g$ of $f'_n$\\
3) Show $f'(x)$ exists for all $x \in \mathbb{R}$ but $f'(0) \neq g(0)$\\
4) Find the values of $x$ such that $f'(x) = g(x)$\\
5) Find the subintervals of $\mathbb{R}$ where $f_n \to f$ uniformly\\
6) Find the subintervals of $\mathbb{R}$ where $f'_n \to g$ uniformly\\

\noindent 1) We break up the pointwise convergence of $f_n$ into the subintervals $(-\infty,0),\{0\},(0,\infty)$. We have $f_n(0) = 0$ for all $n \in \mathbb{N}$ so the limit is $f(0) = 0$. For the subinterval $(0,\infty)$ we have the following 
\[ f_n = \frac{x}{1+ nx^2} \le \frac{x}{nx^2} = \frac{1}{nx}\]
so we have $\lim_{n \to \infty} f_n \to 0$ for $x \in (0,\infty)$. For $x \in (-\infty,0)$ we have \[ \lim_{n \to \infty} f_n(x) = - \lim_{n \to \infty} f_n(-x) \to 0 \] since $-x \in (0,\infty)$.

\noindent 2) We start by calculating $f'_n$ as 
\[ f'_n = \frac{1-nx^2}{(1+nx^2)^2} \]
It is clear from substitution that $f_n(0) = 1$ for all $n \in \mathbb{N}$. The following calculation shows the value of $g$ for $x \in \mathbb{R}$ \textbackslash\{0\}:

\begin{eqnarray*}
\lim_{n \to \infty}  \frac{1-nx^2}{(1+nx^2)^2} &=& \lim_{n \to \infty} \frac{1}{(1+nx^2)^2}  -  \lim_{n \to \infty} \frac{nx^2}{(1+nx^2)^2} \\
&=& \lim_{n \to \infty} \frac{1}{(1+nx^2)^2}  -  \lim_{n \to \infty} \frac{nx^2}{1+2n x^2+n^2 x^4}\\
&=& 0 - 0 = 0
\end{eqnarray*}

\noindent So we have $g(0) = 1$ and $g(x) = 0$ for $x \in \mathbb{R}$ \textbackslash\{0\}.

\noindent 3) Since $f(x) = 0$ for all $x \in \mathbb{R}$ we have $f'(x) = 0$ for all $x$. Because $g(0) = 1$ as was shown in part 2 we have $f'(0) \neq g(0)$.

\noindent 4) $x = 0$ is the only point at which $f'$ and $g$ disagree so $f' = g$ for $x \in \mathbb{R}$ \textbackslash \{0\}.

\noindent 5) We show that the $f_n$ is uniformly convergent from the definition of uniform convergence.\\
\indent First we assume that $|x| < \epsilon$. We have therefore $|x| < \epsilon < \epsilon |1 + n x^2|$ for all $n \in \mathbb{N}$. This gives $|\frac{x}{1+nx^2}| < \epsilon$ for all $n$ so that $N$ may be chosen arbitrarily.\\
\indent Now we assume that $\epsilon \le |x|$. So we have $\frac{1}{|x|} < \frac{1}{\epsilon}$ and if we let $N > \frac{1}{\epsilon^2}$ then for $n > N > \frac{1}{\epsilon^2}$ so $ \frac{1}{n} < \epsilon^2$ and we have
\[ |\frac{x}{1 + nx^2}| < |\frac{x}{nx^2}| < |\frac{1}{nx}| < \frac{1}{n\epsilon} <  \frac{1}{\epsilon} \epsilon^2\ = \epsilon \]
This shows for all $\epsilon$ if we pick $N > \frac{1}{\epsilon^2}$ then for $n > N$ it is true that 
$|\frac{x}{1+nx^2}| < \epsilon$ so $f_n \to 0$ uniformly.\\
6) From part 2 we have the result that $f_n' = \frac{1- nx^2}{(1+nx^2)^2}$. Since $f_n'(\frac{1}{\sqrt{2n}}) = \frac{2}{9}$ it is clear that any interval that has $x = 0$ is a limit point will not be uniformly convergent.\\
\indent We now suppose that $0 < \epsilon \le x$. If we let $N  > \frac{1}{\epsilon^3}$ then for $n > N > \frac{1}{\epsilon^3}$ we have the following calculation:
\[ |\frac{1-nx^2}{(1+nx^2)^2}| <  |\frac{1 + nx^2}{(1+nx^2)^2}| = |\frac{1}{1 + nx^2}| < |\frac{1}{nx^2}| <  \frac{1}{n\epsilon^2} < \frac{1}{\epsilon^2} \epsilon^3 = \epsilon \]
This shows that $f_n' \to g$ uniformly for any interval which does not have zero as limit point

\vspace{15pt}
\begin{flushleft} 
\textbf{Class 18.100B} - Problem 3\\
\rule{500pt}{1pt}\\
\end{flushleft} 
We would like to show that if $f_n \to f$ uniformly and each $f_n$ is bounded then $f_n$ is uniformly bounded and $f$ is bounded. We also will provide an example that $f$ need not be bounded if $f_n$ is not uniformly convergent.\\
\indent By the cauchy characterization of uniform convergence there exists an $N$ such that 
\[ | f_n(x) - f_m(x) | < \epsilon \text{ for n,m $ > N$ and $x \in E$} \] 
If we fix $m = N+1$ and $\epsilon$ and use the triangle equality we get 
\begin{equation}
|f_n(x)| - |f_m(x)| < \epsilon \implies |f_n(x)| < |f_m(x)| + \epsilon \label{eqn:cauchybound}
\end{equation}
Because $f_m$ is bounded we have $|f_m| < M$ so $|f_n(x)| < M + \epsilon$ by ($\ref{eqn:cauchybound}$) for all $n > N$. Because $f_n$ is bounded for each $n$ we choose $B =$ Max$\{B_1,B_2,\cdots,B_{N-1},B_{N},M + \epsilon\}$ where $B_1$ is the bound for $f_1$, $B_2$ for $f_2$ and so on. It is clear from construction then that $|f_n| < B$ for $n \in \mathbb{N}$ which is the definition of uniform boundedness. \\ 
\indent Because $f_n$ is uniformly convergent we fix $\epsilon$ and the following calculation shows that $f$ is bounded:
\[ \text{For $n > N $} |f-f_n| < \epsilon \implies |f|-|f_n| < \epsilon \implies |f| < \epsilon + |f_n| < \epsilon + B \]
\indent Finally, if we let $f_n(x) = \frac{1}{x + \frac{1}{n}}$ on $(0,1)$ then $\lim_{n \to \infty} f_n = \frac{1}{x}$ which is clearly unbounded on $(0,1)$. It is also clear that $f_n$ converges point wise and that $f_n < n$ for all $n \in \mathbb{N}$. This shows that we can have bounded functions converge to an unbounded function if we do not have uniform convergence.
 
\vspace{15pt}
\begin{flushleft} 
\textbf{Class 18.100B} - Problem 4\\
\rule{500pt}{1pt}\\
\end{flushleft} 

We would like to show first that if $f_n \to f$ and $g_n \to g$ uniformly on $E$ then $f_n + g_n \to f + g$ uniformly on $E$. We also want to show that if each $f_n$ and each $g_n$ is bounded on $E$ then $f_n g_n \to fg$ uniformly.
Since $f_n$ and $g_n$ converge uniformly then there exists $N_1$ and $N_2$ such that $|f_n - f | < \frac{\epsilon}{2}$ and $|g_n - g | < \frac{\epsilon}{2}$ for $n > N_1$ and $n > N_2$ respectively. We let $N = $ Max$\{N_1,N_2\}$ and get 
\[ |f_n + g_n - (f+g)| = |f_n - f + g_n - g| < |f_n - f| + |g_n - g| < \frac{\epsilon}{2} + \frac{\epsilon}{2} = \epsilon \]
This shows the first part.\\
\indent Since we are given that $f_n$ and $g_n$ are bounded for each $n$ as well as being uniformly convergent we can apply the result of problem 3 to conclude that both $f_n$ and $g_n$ are uniformly bounded. So we have $|g_n| < G$ for $n \in \mathbb{N}$. We also know from the previous problem that $f$ is bounded so we have $|f| < F$ for all $x \in E$. We perform the calculation as follows. Since $f_n$ and $g_n$ are uniformly convergent there exists $N_1$ and $N_2$ respectively such that for $n > N_1,N_2$
\begin{eqnarray*}
|f_n - f| &\le& \frac{\epsilon}{2G} \\
|g_n - g| &\le& \frac{\epsilon}{2F}
\end{eqnarray*}
We take $N = $ Max$\{ N_1, N_2\}$ so that both inequalities hold and calculate
\begin{eqnarray*}
 |f_n g_n - f g| &=& |f_n g_n - f g_n + f g_n - f g| \\
 		       &=& |g_n(f_n - f) + f (g_n - g)| \\ 
		       &\le&  |g_n||f_n - f| + |f||g_n-g| \\
		       &<& G|f_n - f| + F |g_n-g| \\
		       &\le& G \frac{\epsilon}{2G} + F \frac{\epsilon}{2F} = \epsilon
\end{eqnarray*}
So $ |f_n g_n - f g| < \epsilon $ for all $n > N$ and $f_n g_n \to fg$ uniformly.

\vspace{15pt}
\begin{flushleft} 
\textbf{Class 18.100B} - Problem 5\\
\rule{500pt}{1pt}\\
\end{flushleft} 

We are given the following sequences 
\begin{eqnarray*}
 f_n(x) &=& x(1 + \frac{1}{n}) \\
 g_n(x) &=& \left\{
	\begin{array}{ll}
		\frac{1}{n}  & \mbox{if } x \in \mathbb{R} \text{\textbackslash} \mathbb{Q} \\
		q + \frac{1}{n} & \mbox{if } x \in \mathbb{Q} \text{ and } x = \frac{p}{q}
	\end{array}
\right.
\end{eqnarray*}
It is clear that the point wise limits of these sequences are 
\begin{eqnarray*}
 f(x) &=& x\\
 g(x) &=& \left\{
	\begin{array}{ll}
		0 & \mbox{if } x \in \mathbb{R} \text{\textbackslash} \mathbb{Q} \\
		q & \mbox{if } x \in \mathbb{Q} \text{ and } x = \frac{p}{q}
	\end{array}
\right.
\end{eqnarray*}

For any bounded interval $[a,b]$ we have $x \le b$. This allows us to chose $N > \frac{b}{\epsilon}$ so that if $n > N$ we have
\[ |f_n(x) - f(x) | < \frac{x}{n} < \frac{b}{n}  < b \frac{\epsilon}{b} = \epsilon \]
Similarly we choose $N > \frac{1}{\epsilon}$ so if  $n > N$ to show that 
\[ |g_n(x) - g(x)| < \frac{1}{n} < \epsilon \]
so that both $f_n \to f$ and $g_n \to g$ uniformly.
Now consider the following function:
\begin{eqnarray*}
|f_n(x)g_n(x) - f(x)g(x)| &=& \left\{
	\begin{array}{ll}
		| \frac{x}{n}(1 + \frac{1}{n}) | & \mbox{if } x \in \mathbb{R} \text{\textbackslash} \mathbb{Q} \\
		| \frac{xq}{n} + \frac{x}{n} + \frac{x}{n^2} | & \mbox{if } x \in \mathbb{Q} \text{ and } x = \frac{p}{q}
	\end{array}
\right.
\end{eqnarray*}

If we let $B = $ Min $\{ |a|,|b| \}$ then we get 
\begin{eqnarray}
| \frac{xq}{n} + \frac{x}{n} + \frac{x}{n^2} | &\ge& |\frac{Bq}{n} + \frac{B}{n} + \frac{B}{n^2}| \nonumber \\
 \implies | \frac{xq}{n} + \frac{x}{n} + \frac{x}{n^2} | &>& |\frac{Bq}{n}| \label{eqn:lowbound}
\end{eqnarray}

By the density of the rationals we know that on any given interval $[a,b]$ there is an $r \in \mathbb{Q}$ such that $r \in (a,b)$. We chose such a $r$ and note that since $lim_{n \to \infty} r + \frac{1}{n} = r$ we pick $\epsilon = $ Min $\{ |r-a|, |r-b| \}$ and there is an $N$ such that for $n > N$ we have $r + \frac{1}{n} \in [a,b]$. This is to say that after a certain point $N$ we can be sure that all the elements of the sequence $r-\frac{1}{n}$ are contained in the interval.\\
\indent We let $r = \frac{p'}{q'}$ which gives for the sequence $r - \frac{1}{n} = \frac{np' - q'}{q'n}$. So if we pick $N_1 = $ Max$\{n,N\}$ then we have $q =  q'n$ in ($\ref{eqn:lowbound}$) which gives
\[ | \frac{xq}{n} + \frac{x}{n} + \frac{x}{n^2} | > |\frac{Bq'N_1}{n}| > Bq' \quad \text{for all $n \in \mathbb{N}$ }\]
This shows that $f_n g_n$ does not converge uniformly.

\vspace{15pt}
\begin{flushleft} 
\textbf{Class 18.100B} - Problem 6\\
\rule{500pt}{1pt}\\
\end{flushleft} 

We would like to show that $g \circ f_n \to g \circ f$ uniformly given that $f_n \to f$ uniformly and $f_n$ is uniformly bounded.\\
Since $f_n$ is uniformly convergent, for any $\delta$ there exists an $N$ such if $n > N$ then 
\[  |f_n(x) - f(x)| < \delta \text{ for all $x \in E$} \]
Let $M$ be a uniform bound of $f_n$ then note that because $g$ is continuous on $[-M,M]$ it is uniformly continuous. By the uniform continuity of $g$ we know that for a given $\epsilon$ that if $|f_n(x) - f(x)| < \delta $ and $f_n(x),f(x) \in [-M,M]$ then it is true that 
\[ |g(f_n(x)) - g(f(x))| < \epsilon \]
This says that for all $\epsilon$ there exists an $N$ such that if $x \in E$ and $n > N$ then 
\[ |g \circ f_n(x) - g \circ f(x)| < \epsilon \]
this says that $g \circ f_n \to g \circ f$ uniformly.

\vspace{15pt}
\begin{flushleft} 
\textbf{Class 18.100B} - Problem 7\\
\rule{500pt}{1pt}\\
\end{flushleft} 

\noindent We want to show \\
a) If $P_n(x)$ is defined as
\begin{eqnarray*}
P_0(x) &=& 0 \\
P_{n+1}(x) &=& P_n(x) + \frac{1}{2} (x - P_n^2(x))
\end{eqnarray*}
then $P_n(x) \to \sqrt{x}$ uniformly. \\

b) There exists a sequence of polynomials that converges uniformly on $[-1,1]$ to $f(x) = |x|$

For part a) first we note that $P_n$ defines a sequence of polynomials. This is true since as a basis for induction $P_0$ is a trivial polynomial. For the induction step we note that because a product of a polynomial and a polynomial is a polynomial (by the binomial theorem) then $P_n^2$ is also a polynomial. This means the expression $P_n(x) + \frac{1}{2}x-\frac{1}{2}P_n^2(x)$ is a sum of polynomials showing that $P_{n+1}$ is a polynomial.\\
\indent So $P_n$ is a sequence of continuous function that we now show converges point wise.\\
\indent We show $\sqrt{x} \ge P_n(x)$ for $ x \in [0,1]$ inductively. The inequality is trivially true for $\sqrt{x} \ge 0 = P_0(x)$. Now assume that $\sqrt{x} \ge P_n(x)$. This allows us to assume that
\begin{equation}
\sqrt{x} - P_n(x) \ge 0 \label{eqn:firstpolyineq}
\end{equation}
and also
\begin{eqnarray}
\sqrt{x} \ge P_n(x) &\implies& \sqrt{x} + \sqrt{x} \ge \sqrt{x} + P_n(x) \nonumber \\
			    &\implies& 2 \sqrt{x} \ge \sqrt{x} + P_n(x) \nonumber \\
			    &\implies& \sqrt{x} \ge \frac{1}{2}( \sqrt{x} + P_n(x) ) \nonumber \\
			    &\implies& \sqrt{x} - \frac{1}{2}( \sqrt{x} + P_n(x) ) \ge 0 \nonumber\\
			    &\implies& 1 - \frac{1}{2}( \sqrt{x} + P_n(x) ) \ge 0 \quad \text{since $x \in$ [0,1] $\implies 1 \ge \sqrt{x}$ } \label{eqn:secondpolyineq}
\end{eqnarray}

\noindent Now we perform the following calculation:
\begin{eqnarray*}
\sqrt{x} - P_{n+1}(x) &=& \sqrt{x} - P_n(x) - \frac{1}{2}(x - P_n^2(x))\\
			       &=& \sqrt{x} - P_n(x) - \frac{1}{2}(\sqrt{x} - P_n(x))(\sqrt{x} + P_n(x))\\
			       &=& (\sqrt{x} - P_n(x))(1 - \frac{1}{2}(\sqrt{x} + P_n(x)))\\
			       &\ge& 0 \quad \text{by equations (\ref{eqn:firstpolyineq}) and (\ref{eqn:secondpolyineq})}
\end{eqnarray*}
This shows that $\sqrt{x} \ge P_{n+1}(x)$ which completes the induction.\\
Now since $\sqrt{x} \ge P_n(x)$ for all $n$ we have $x > P_n^2(x) \implies x - P_n^2(x) \ge 0$ for all $n$.  This allows us to show that $P_n$ is increasing since 
\begin{eqnarray*}
&& P_{n+1}(x) = P_n(x) + \frac{1}{2} (x - P_n^2(x)) \\
&\implies& P_{n+1}(x) - P_n(x) = \frac{1}{2} (x - P_n^2(x))  \\ 
&\implies & P_{n+1}(x) - P_n(x) \ge 0 \quad \text{by the previous comments}
\end{eqnarray*}
We have shown that for a give $x$ $P_n(x)$ is increasing and bounded. By a theorem in rudin this means that $P_n$ converges. This allows to calculate as follows which demonstrates the point wise convergence
\begin{eqnarray*}
&& \lim_{n \to \infty} P_{n+1}(x) = \lim_{n \to \infty} P_n(x) + \frac{1}{2} (x - \lim_{n \to \infty} P_n^2(x))\\
&\implies& P(x) = P(x) + \frac{1}{2} (x - P^2(x)) \\
&\implies& 0 = \frac{1}{2} (x - P^2(x)) \\ 
&\implies& \sqrt{x} = P(x)
\end{eqnarray*}

\indent We now briefly consider the set of polynomials defined by $P'_n(x) = 1 - P_n(x)$. Since $P_n$ converges point wise so does $P'_n$ by a basic property of limits. However since $P_{n+1}(x) \ge P_n(x)$ we have \\$P'_{n+1}(x) = 1 - P_{n+1}(x) \le 1 - P_n(x) = P'_n(x)$. This shows that $P'_n$ is a decreasing sequence. By a theorem in rudin, if the functions in a sequence of functions on a compact set are continuous, limit to a continuous function and are decreasing then the sequence uniformly converges. So $\lim P'_n \to 1 - \sqrt{x}$ uniformly but then for $\epsilon$ there exists $N$ such that if $n > N$ we have 
\begin{eqnarray*} 
&& |P'_n(x) - (1 - \sqrt{x})| < \epsilon \\
&\implies& |1 - P_n - (1 - \sqrt{x})| < \epsilon \\
&\implies& |P_n - \sqrt{x}| < \epsilon
\end{eqnarray*}
so $P_n \to \sqrt{x}$ uniformly. \\

\noindent For part b) we note that the function $f(x) = x^2$ is such that $f : [-1,1] \to [0,1]$. By part a the sequence of polynomials $P_n(x) \to \sqrt{x}$ uniformly. Since $P_n(x)$ is a polynomial for all $n$ so is $P_n(x^2)$. So we have $P_n \circ f : [-1,1] \to [0,1]$ and $P_n \circ f(x) = P_n(f(x)) = P_n(x^2) \to \sqrt{x^2} = |x|$ which satisfies the required conditions.



\end{document}  